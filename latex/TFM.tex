\documentclass[a4paper,12pt]{report}
\usepackage{graphicx}
\usepackage{cancel}	%para cancelar expresiones en fórmulas
\usepackage{multicol}
\usepackage{multirow}
\usepackage{amsmath}
\usepackage{float}
\usepackage{url}
\usepackage{color}
\usepackage{tablefootnote}
\usepackage[12pt]{moresize}	%más tamaños de letra

\usepackage{siunitx}	%support for units, numbers
\sisetup{output-decimal-marker={.}, separate-uncertainty}

\usepackage{hyperref}
\hypersetup{
	colorlinks=true,
	linkcolor=blue,
	citecolor=red,
	bookmarksopen=true,
	bookmarksopenlevel=1,
	urlcolor=blue
}

\usepackage{tikz}	%%TIKZ for drawings, shapes
\usetikzlibrary{positioning}
\tikzset{
	nodo/.style={
		rectangle,
		draw,
		black,
		thick,
		rounded corners,
		node distance=1.5cm,
		inner sep=0.3cm,
		font=\large
	}
}

%%% NOMBRES DE COSAS
\renewcommand{\abstractname}{Abstract}
\renewcommand{\figurename}{Figure}
\renewcommand{\tablename}{Table}
\renewcommand{\contentsname}{Table of Contents}
\renewcommand{\appendixname}{Appendix}
\renewcommand{\bibname}{References}
\renewcommand{\chaptername}{Part}

%%% COMANDOS COMUNES
\newcommand{\dif}{\text{d}}
\newcommand{\ddt}[1]{\frac{\dif #1}{\dif t}}
\newcommand{\der}[2]{\frac{\dif #1}{\dif #2}}
\newcommand{\mder}[3]{\frac{\dif^{#3}#1}{\dif #2^{#3}}}		%derivada múltiple
\newcommand{\norm}[1]{\left\| #1 \right\|}
\newcommand{\an}{($\alpha$,n) }

%%%%%%%%%%%%%%%%%%%%%%%%%%%%%%%%%%%%%%%%%%%%%%%%%%%%%%%%%
\usepackage[a4paper]{geometry}
\geometry{top=2cm,bottom=2cm,left=1.8cm,right=1.8cm}

\author{Erik Cárdenas Mayoral}
\title{\HUGE{Title}}
\date{date}
\makeindex
%%%%%%%%%%%%%%%%%%%%%%%%%%%%%%%%%%%%%%%%%%%%%%%%%%%%%%%%%
\begin{document}
\maketitle

\begin{abstract}
Abstract
\end{abstract}

\tableofcontents

\chapter{Introduction and motivation}
Explain what \an reactions are.

Relevance of \an reactions for: dark matter search, astrophysics.

The objective of the measurements at CNA are to check the viability for carrying out more \an measurements.


\chapter{Experimental setup and procedure}
\section{Setup}
We do the measurements for 27Al because it is the standard, with the best available measurements to compare to.

Tandem accelerator.
Difficulty of accelerating alphas with the reactor, with it being designed for deuterons.

Distances and angles of all relevant detectors.

LaBr detectors for gammas.
MONSTER detector for neutrons.

\section{Procedure}
Explain the procedure for activation measurements.

Explain the procedure for pulsed measurements.


\chapter{Measurements and results}


\chapter{Conclusions}
We have managed to measure reaction yield with X uncertainty (given different detectors, days or months) and X agreement with previous measurements.
We have managed to measure the neutron energy spectra with X uncertainty/resolution and X agreement with previous measurements.

\end{document}
