\documentclass[a4paper,12pt]{report}
\usepackage{graphicx}
\usepackage{cancel}	%para cancelar expresiones en fórmulas
\usepackage{multicol}
\usepackage{multirow}
\usepackage{amsmath}
\usepackage{float}
\usepackage{url}
\usepackage{color}
\usepackage{tablefootnote}
\usepackage[12pt]{moresize}	%más tamaños de letra
\usepackage{bm}	%bold para math

\usepackage{titlesec}
\titleformat{\chapter}[display]
{\normalfont\bfseries}{}{0pt}{\Huge}	%quitar header de chapter

\usepackage{siunitx}	%support for units, numbers
\sisetup{output-decimal-marker={.}, separate-uncertainty}

\usepackage{hyperref}
\hypersetup{
	colorlinks=true,
	linkcolor=blue,
	citecolor=red,
	bookmarksopen=true,
	bookmarksopenlevel=1,
	urlcolor=blue
}

\usepackage{tikz}	%%TIKZ for drawings, shapes
\usetikzlibrary{positioning}
\tikzset{
	nodo/.style={
		rectangle,
		draw,
		black,
		thick,
		rounded corners,
		node distance=1.5cm,
		inner sep=0.3cm,
		font=\large
	}
}

%%% NOMBRES DE COSAS
\renewcommand{\abstractname}{Abstract}
\renewcommand{\figurename}{Figure}
\renewcommand{\tablename}{Table}
\renewcommand{\contentsname}{Table of Contents}
\renewcommand{\appendixname}{Appendix}
\renewcommand{\bibname}{References}
\renewcommand{\chaptername}{Part}

%%% COMANDOS COMUNES
\newcommand{\dif}{\text{d}}
\newcommand{\ddt}[1]{\frac{\dif #1}{\dif t}}
\newcommand{\der}[2]{\frac{\dif #1}{\dif #2}}
\newcommand{\mder}[3]{\frac{\dif^{#3}#1}{\dif #2^{#3}}}		%derivada múltiple
\newcommand{\norm}[1]{\left\| #1 \right\|}
\newcommand{\an}{($\alpha$,n) }
\newcommand{\Aliso}{\textsuperscript{27}Al }
\newcommand{\Piso}{\textsuperscript{30}P }
\newcommand{\Na}{\textsuperscript{22}Na }

%%%%%%%%%%%%%%%%%%%%%%%%%%%%%%%%%%%%%%%%%%%%%%%%%%%%%%%%%
\usepackage[a4paper]{geometry}
\geometry{top=2cm,bottom=2cm,left=1.8cm,right=1.8cm}

\makeindex
%%%%%%%%%%%%%%%%%%%%%%%%%%%%%%%%%%%%%%%%%%%%%%%%%%%%%%%%%
\begin{document}
\begin{titlepage}
	\centering
	\Huge Máster interuniversitario en física nuclear\par
	\vspace*{3cm}
	\HUGE \textbf{Measurements of ($\bm{\alpha}$,n) reactions in CNA/HISPANOS}\par	%TBD? más largo?
	\vspace{1cm}
	\includegraphics[width=0.35\textwidth]{us.png}\\
	\vspace{1cm}
	\Large \textsc{Erik Cárdenas Mayoral}\par
	\vspace{2cm}
	Tutores:\\
	Carlos Guerrero Sánchez\\
	María Begoña Fernández Martínez\par
	\vfill
	00/06/2023
\end{titlepage}

\begin{abstract}
Abstract
\end{abstract}

\tableofcontents

\chapter{Introduction and motivation}
·Explain what \an reactions are.\\

·Relevance of \an reactions for: dark matter search, astrophysics.\\

·The objective of the measurements at CNA are to check the viability for carrying out more \an measurements.\\


\chapter{Experimental setup and procedure}
\section{Setup}
\subsection{Tandem 3MV accelerator}
·Explain how a tandem accelerator works.\\
·Explain pulsed mode: chopper, buncher.\\
·Difficulty of using alphas with the accelerator, with it being designed for deuterons. Graph of gamma flash highlighting assymetry and tail.\\

·Explain the different parts of the neutron line. Photos.\\

\subsection{Target}
·We do the measurements for \Aliso because it is the standard, with the best available measurements to compare to.\\
·Explain how we measure the number of incident alphas with the current integrator.\\

\subsection{Radiation detectors}
·Table with distances and angles of all relevant detectors.\\

\subsubsection{LaBr}
·We use LaBr detectors for gammas.\\
·LaBr are inorganic scintillating detectors.\\
·LaBr detectors have a spectroscopic response.\\

·We calibrate them with the \Na \qty{511}{\keV} peak. Calibration spectrum.\\
·Alson \textsuperscript{137}Cs spectrum?\\
·We calculate detector efficiency.\\

\subsubsection{MONSTER}
·We use a MONSTER detector for neutrons.\\
·MONSTER is an organic scintillating detector.\\
·MONSTER can do PSD.\\
·PSD allows us to separate neutrons and gammas. Show PSD separation plot.\\
·Explain PSD.\\
·We use a theoretical curve to calibrate in efficiency. Show curve.\\

\section{Procedure}
\subsection{Activation measurements}
·Explain the procedure for activation measurements:\\
·Beam is continuous.\\
·\Piso created when an \an reaction occurs.\\
·\Piso decays with a X minute half life and emits \qty{511}{\keV} with X intensity.\\
·The \Aliso target is irradiated for several half-lifes, then left to decay.\\
·Show and explain resulting histogram.\\

\subsection{Pulsed beam measurements}
·Explain the procedure for pulsed measurements:\\
·Beam is pulsed.\\
·Alphas arrive all at the same time (lie), \an reactions happen.\\
·Both gammas and neutrons are produced at the same time.	%TBD:mirar tiempo de producción de gammas\\
·Gammas move at c, arrive to detectors d/c time later. This is called gamma flash.\\
·Neutrons have different speed depending on energy, arrive with different ToF.\\
·Show example ToF plot.\\
·Measuring ToF, we measure neutron energy.\\
·We time the pulses by using a signal from the chopper/buncher.	%TBD:chopper o buncher?\\

·Alphas do not arrive all at the same time: X nanosecond wide pulses. This introduces uncertainty in the ToF, and thus energy, measurements.\\
·Pulses are not symmetrical: tails, several peaks. This means we cannot treat pulses as gaussians and the introduced error in ToF as their standard deviation. Show graph of uneven gamma flash.\\


\chapter{Measurements and results}
·Measurements were carried out in February and April.\\

\begin{table}[h]	%Tabla con datos de las medidas de activación
\centering
\begin{tabular}[c]{>{\bfseries}r||c|c|c|c}
	N & Energy (\unit{\keV}) & Current (\unit{\nano\A}) & Activation time (\unit{\s}) & Date\tablefootnote{All took place in 2023.\label{date_tablefootnote}} \\ \hline
	1	&\num{5500}&\num{128}&\num{867}&Feb 22nd\\ \hline
	2	&\num{7000}&\num{101}&\num{981}&Feb 22nd\\ \hline
	3	&\num{8500}&\num{128}&\num{909}&Feb 22nd\\ \hline
	4	&\num{8500}&\num{192}&\num{899}&Feb 23rd\\ \hline
	5	&\num{5500}&\num{123}&\num{605}&Apr 17th\\ \hline
	6	&\num{5500}&\num{312}&\num{599}&Apr 17th\\ \hline
	7	&\num{8250}&\num{193}&\num{452}&Apr 18th\\ \hline
	8	&\num{7000}&\num{216}&\num{466}&Apr 18th\\ \hline
	9	&\num{5500}&\num{151}&\num{451}&Apr 18th\\ \hline
	10	&\num{7500}&\num{183}&\num{448}&Apr 18th\\ \hline
\end{tabular}
\label{activation_measurements_table}
\caption{Activation measurements}
\end{table}

\begin{table}[h]	%Tabla con datos de las medidas de haz pulsado
\centering
\begin{tabular}[c]{>{\bfseries}r||c|c|c}
	N& Energy (\unit{\keV}) & Distance (\unit{\meter}) & Date\footref{date_tablefootnote} \\ \hline	%TBD?incertidumbre en distancia?
	1&\num{5500}&\num{1.0}&April 17th\\ \hline
	2&\num{5500}&\num{1.0}&April 18th\\ \hline
	3&\num{7000}&\num{1.0}&April 18th\\ \hline
	4&\num{8250}&\num{1.0}&April 18th\\ \hline
	5&\num{5250}&\num{2.0}&April 18th\\ \hline
\end{tabular}
\label{pulsed_measurements_table}
\caption{Pulsed measurements}
\end{table}

\section{Activation}
We can describe the number of \Piso nuclei in the target, $N$, using the differential equation:
\begin{equation}
	\ddt{N} = I(t) -\lambda N
\end{equation}
where $\lambda$ is the decay constant of \Piso and $I(t)$ is the number of \Piso nuclei created (by \an reactions) per unit of time.
$I(t)$ will be proportional to the current of alphas hitting the target.
If expressed in number of alphas per second, the proportionality constant will be the yield of the reaction; the number of \an reactions per incident $\alpha$ particle.\\

In order to get $I(t)$, we analyze the activity of \Piso nuclei as they decay by $\beta +$, which will be $A = \lambda N$.
We don't measure the activity directly, but rather the activity times $1.798$: the intensity of the \qty{511}{\keV} emission.	%TBD: referencia para intensidad, tipo de decay

We use three different methods to get $I(t)$, depending on wether $I(t)$ is considered constant during the activation period or not; and on the part of the histogram being looked at.

·Show example activation time histogram.\\

\subsubsection{Decay fit}
After the activation period is over, we are left with a certain number of \Piso nuclei that will decay following a radioactive decay law.
For the \textit{decay} fit, we fit this part of the activation histogram to a negative exponential, $A(t) = A\textsubscript{EOA} e^{-\lambda t}$, (plus a background) to get the \Piso activity at the end of activation, $A$\textsubscript{EOA}.
We divide $A$\textsubscript{EOA} by $\lambda$ to get the number of \Piso nuclei at the end of activation, $N$\textsubscript{EOA}.
·Show decay fit.\\

We can solve the differential equation during the activation period analytically by taking $I(t)$ as a constant, $I$.
We get
\[ \ddt{N} = I -\lambda N  \]
and its solution is:
\begin{equation}
	N(t) = \frac{I}{\lambda} + \left(  N_0 - \frac{I}{\lambda}  \right) e^{-\lambda t}
\end{equation}
with $N_0$ the initial number of \Piso nuclei, assumed to be \num{0}.

$N$\textsubscript{EOA} will then be equal to:
\[ N\textsubscript{EOA} =N(\Delta t) = \frac{I}{\lambda} - \frac{I}{\lambda} e^{-\lambda \Delta t} = \frac{I}{\lambda} \left(1 - e^{-\lambda \Delta t} \right)\Longrightarrow A\textsubscript{EOA}=I\left(1-e^{-\lambda\Delta t}\right)\Longrightarrow \]
\begin{equation}
	I = \frac{A\textsubscript{EOA}}{1 - e^{-\lambda \Delta t}}
\end{equation}
with $\Delta t$ the total activation time.
As $I$ is the number of \Piso created per second, we can get the yield of the reaction by dividing over the number of incident alpha particles per unit of time:
\begin{equation}
	\text{yield} = \frac{I}{\text{alphas per second}} = \frac{A\textsubscript{EOA}}{\text{alphas per second} \cdot \left(1 - e^{-\lambda \Delta t} \right)}
\end{equation}

All that is left is to correct for the efficiency of the detector, and the final formula for the yield is:
\begin{equation}
	\text{yield} = \frac{\text{fit result}}{\text{alphas per second} \cdot \text{detector efficiency} \cdot \left(1 - e^{-\lambda \Delta t} \right) \cdot 1.798}	%TBD? use symbols instead?
\end{equation}
%TBD: mention binwidth

·Graph with results compared to EXFOR.\\

\subsubsection{Unified and rise fit}
·We fit to a function that integrates the diffeq, taking into account the current step by step, a constant background, a background proportional to current, and an initial number of \Piso nuclei.\\
·We fit either the whole time histogram (unified) or only the activation (rise). Show fits.\\
·The fit returns the number of \an reactions per unit of charge.\\
·We convert from charge to number of alphas.\\
·We correct for efficiency.\\

·Problems with loss of current data in April.\\
·Problems with DAQ loss in February. Show time histogram.\\

·Graphs with results compared to EXFOR.\\

\section{Pulsed}
·We separate gamma flash and neutron response. Show tof histograms.\\
·Explain deconvolution:\\
·We fit the neutron response histogram to a function that takes as input the gamma flash histogram and 50 parameters corresponding to the neutron spectra in time of flight. The function loops through the gamma histogram to add the contribution of each bin.\\
·The result is to remove the effect of the width and assymetry of the gamma flash. Show plot of neutron response+response to delta.\\

·We convert the resulting parameters from time of flight to energy. Show energy spectra plot.\\
·We correct considering the efficiency of the detector. Show final plot.\\


\chapter{Conclusions}
·We have managed to measure \an reaction yield in \Aliso with X uncertainty (given different detectors, days or months) and X agreement with previous measurements.\\
·We have managed to measure the corresponding neutron energy spectra with X uncertainty/resolution and X agreement with previous measurements.\\

\end{document}
